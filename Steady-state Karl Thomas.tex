

\documentclass{article}

%%%Packages
\usepackage[utf8]{inputenc}
\usepackage{amsmath, amsthm, amssymb, amsfonts, dsfont, mathrsfs, dirtytalk, tikz-cd, adjustbox, url, nicefrac, booktabs,array,longtable, tabu,bm, hyperref, graphicx, geometry,comment}
\usepackage{caption}

\usepackage[T1]{fontenc}    
%\usepackage[margin=1in]{geometry}
\usepackage[english]{babel}
%\usepackage[pagebackref=true]{hyperref}
%\usepackage{hyperref}
\usepackage[capitalise]{cleveref}
%\usepackage{arxiv}

%%%Backreferences
%\renewcommand*{\backref}[1]{}
%\renewcommand*{\backrefalt}[4]{%
%    \ifcase #1%
%          \or (p.~#2.)%
 %         \else (p.~#2.)%
  %  \fi%
   % }

%%%%Macros
\newcommand{\e}{\epsilon}
\newcommand{\f}{\forall}
\newcommand{\R}{\mathbb R}
\newcommand{\st}{\text{ s.t. }}
\newcommand{\ex}{\exists}
\newcommand{\p}{\mathbb P}
\newcommand{\D}{\mathcal D(\Omega)}
\newcommand{\N}{\mathbb N}
\newcommand{\oo}{\Omega}
\newcommand{\harpoon}{\rightharpoondown}
\newcommand{\E}{\mathbb E}


%%%%Theorem environments%%%%
\newtheorem{theorem}{Theorem}
\newtheorem{lemma}[theorem]{Lemma}
\newtheorem{proposition}[theorem]{Proposition}
\newtheorem{example}[theorem]{Example}
\newtheorem{corollary}[theorem]{Corollary}
\newtheorem{remark}[theorem]{Remark}
\newenvironment{claim}[1]{\par\noindent\emph{Claim:}\space#1}{}
\newenvironment{proofclaim}[1]{\par\noindent\textit{Proof of claim.}\space#1}{\hfill $\blacksquare$}
\newenvironment{sketchproof}[1]{\par\noindent\textit{Sketch proof.}\space#1}{\hfill $\square$}

%% Use the option review to obtain double line spacing
%% \documentclass[authoryear,preprint,review,12pt]{elsarticle}

%% Use the options 1p,twocolumn; 3p; 3p,twocolumn; 5p; or 5p,twocolumn
%% for a journal layout:
%% \documentclass[final,1p,times,authoryear]{elsarticle}
%% \documentclass[final,1p,times,twocolumn,authoryear]{elsarticle}
%% \documentclass[final,3p,times,authoryear]{elsarticle}
%% \documentclass[final,3p,times,twocolumn,authoryear]{elsarticle}
%% \documentclass[final,5p,times,authoryear]{elsarticle}
%% \documentclass[final,5p,times,twocolumn,authoryear]{elsarticle}

%% For including figures, graphicx.sty has been loaded in
%% elsarticle.cls. If you prefer to use the old commands
%% please give \usepackage{epsfig}

%% The amssymb package provides various useful mathematical symbols
%\usepackage{amssymb}
%% The amsthm package provides extended theorem environments
%% \usepackage{amsthm}

%% The lineno packages adds line numbers. Start line numbering with
%% \begin{linenumbers}, end it with \end{linenumbers}. Or switch it on
%% for the whole article with \linenumbers.\usepackage{lineno}

%\journal{Journal of Mathematical Psychology}

\begin{document}


\begin{lemma}[Steady-state]
\label{lemma:steady-state}
The system reaches steady-state: 
\begin{equation*}
    D_{KL}[Q(s_\tau, A) ||P(s_\tau, A)]=0,
\end{equation*}
 if and only if, the surprisal over policies $-\log Q(\pi)$ and the Gibbs energy $G(\pi; \beta)$, are equal when averaged under $Q$
\begin{align*}
    \mathbb E_Q[-\log Q(\pi)]&= \mathbb E_Q[G(\pi; \beta)],
\end{align*}
where, up to a constant,
\begin{align*}
    G(\pi; \beta) &= D_{KL}[Q(s_\tau, A|\pi)||P(s_\tau,A)]-\mathbb E_{Q(o_\tau, s_\tau, A|\pi)} [\beta \log P(o_\tau |s_\tau, A)] \\
    \beta &:= \frac{\mathbb E_Q[\log Q(\pi|s_\tau,A)]}{\mathbb E_Q[\log P(o_\tau|s_\tau,A)]} \geq 0.
\end{align*}
\end{lemma}

\begin{proof}
Let us unpack the Gibbs energy expected under $Q$:
\begin{equation*}
    \begin{split}
        \mathbb E_Q[G(\pi; \beta)] &=\mathbb E_Q[D_{KL}[Q(s_\tau, A|\pi)||P(s_\tau,A)]-\mathbb E_{Q(o_\tau, s_\tau, A|\pi)} [\beta \log P(o_\tau |s_\tau, A)]] \\
        &= \mathbb E_Q[D_{KL}[Q(s_\tau, A|\pi)||P(s_\tau,A)]]\\
        &-\frac{\mathbb E_Q[\log Q(\pi|s_\tau,A)]}{\mathbb E_Q[\log P(o_\tau|s_\tau,A)]} \mathbb E_Q [\mathbb E_{Q(o_\tau, s_\tau, A|\pi)} [ \log P(o_\tau |s_\tau, A)]] \\
        &= \mathbb E_Q[\log Q(s_\tau, A|\pi)-\log P(s_\tau, A)-\log Q(\pi |s_\tau,A)] \\
        &= \mathbb E_Q[-\log Q(\pi)-\log P(s_\tau, A)+\log Q(s_\tau,A)] \\
        &= \mathbb E_Q[-\log Q(\pi)] +D_{KL}[Q(s_\tau,A)||P(s_\tau, A)].
    \end{split}
\end{equation*}
Therefore,
\begin{equation*}
    \mathbb E_Q[G(\pi; \beta)] = \mathbb E_Q[-\log Q(\pi)] \iff D_{KL}[Q(s_\tau,A)||P(s_\tau, A)]=0.
\end{equation*}
\end{proof}

\begin{corollary}
Each approximate posterior
\begin{equation*}
    \label{eq:posterior_gibbs}
   Q(\pi) =\sigma(-G(\pi; \beta)),\quad \beta \geq 0
\end{equation*}
\textbf{does not} necessarily cause the system to reach steady-state.
\end{corollary}

\begin{proof}
Let $\beta \geq 0$.
\begin{equation*}
   Q(\pi) = \frac{1}{Z}\exp(-G(\pi; \beta))
\end{equation*}
where $Z:= \sum_{\pi} \exp(-G(\pi; \beta))$. Then
\begin{equation*}
    -\log Q(\pi) = G(\pi; \beta) + \log Z
\end{equation*}
Following with the above proof, we obtain 
\begin{equation*}
    D_{KL}[Q(s_\tau, A) ||P(s_\tau, A)]= \log Z.
\end{equation*}
By non-negativity ok the KL divergence $Z \geq 1$. $Z=1$ corresponds to reaching steady-state. Minimising $Z$ corresponds to getting closer to steady-state.
\end{proof}

What is also odd is that to minimise $Z= \sum_{\pi} \exp(-G(\pi; \beta))$ it suffices to maximise $G(\pi)?$ 

\end{document}